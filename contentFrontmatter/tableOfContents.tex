%-------------------------------------------------------------------------------
%-------------------------------------------------------------------------------
% # Inhalt des Inhaltsverzeichnisses (engl. table of contents).
% 	Optionen für das Inhaltsverzeichnis.
%	Häufig findet man nach dem Inhaltsverzeichnis auch noch die Verzeichnisse  
%	der Gleitumgebungen, beispielsweise von Tabellen und Abbildungen.
%
%-------------------------------------------------------------------------------
%-------------------------------------------------------------------------------

%-------------------------------------------------------------------------------
% Option: toc=[listof|listofnumered|nolistof] 
%
% * listof:
%		Aufnahme des Abbildungs- und Tabellenverzeichnisses ins 
%		Inhaltsverzeichnis (Wenn diese am Buchende stehen).
%		Dabei werden auch Verzeichnisse berücksichtigt, die mit Hilfe des 
%		"float"-Pakets oder "floatrow" erstellt werden.
% * listofnumered:
%		Diese Verzeichnisse erhalten Grundsätzlich keine Kapitelnummern.
%		"listofnumbered" hebt das auf.
% * nolistof:
%		Keine Aufnahme des Abbildungs- und Tabellenverzeichnisses ins 
%		Inhaltsverzeichnis (Wenn diese am Buchende stehen).
%		Dabei werden auch Verzeichnisse berücksichtigt, die mit Hilfe des 
%		"float"-Pakets oder "floatrow" erstellt werden.
%
%-------------------------------------------------------------------------------

% \KOMAOption{toc}{listof}

%-------------------------------------------------------------------------------
% Option: toc=[index|noindex]  
%
% * index:
%		Aufnahme des Stichwortverzeichnisses ins Inhaltsverzeichnis.
%		Eine Kapitelnummer für dieses Verzeichnis wird nicht unterstützt.
% * noindex:
%		Keine Aufnahme des Stichwortverzeichnisses ins Inhaltsverzeichnis.
%
%-------------------------------------------------------------------------------

% \KOMAOption{toc}{index}

%-------------------------------------------------------------------------------
% Option: toc=[bibliography|bibliographynumbered|nobibliography]  
%
% * bibliography:
%		Eintragung des Literaturverzeichnis ins Inhaltsverzeichnis.
% * bibliographynumbered:
%		Vergabe einer Kapitelnummer für das Literaturverzeichnis und 
%		Eintragung ins Inhaltsverzeichnis.
% * nobibliography:
%		Keine Eintragung des Literaturverzeichnis ins Inhaltsverzeichnis.
%
%-------------------------------------------------------------------------------

% \KOMAOption{toc}{listof}

%-------------------------------------------------------------------------------
% Option: toc=[ graduated | indent | indented | flat| left]
%
% * [graduated | indent | indented]: 
%		Normalerweise wird das Inhaltsverzeichnis so formatiert, dass die 
%		Gliederungsebenen untersciedlich weit eingeszogen werden.
% * [flat | left]:
%		Werden viele Gliederungspunkte verwendet, so werden Gliederungsnummern
%		sehr breit - der vorgesehen Platz reicht nicht aus.
%		"flat" schaltet den Unterschiedlichen Einzug der Gliederungspunkte aus.
%
%-------------------------------------------------------------------------------

% \KOMAOption{toc}{graduated}

%-------------------------------------------------------------------------------
% Option: toc=[numberline|nonumberline]
%
% * numberline:
%	Die Eigenschaft "numberline" wird für das Inhaltsverzeichnis gesetzt.
%	Dadurch werden nicht nummerierte Einträge linksbündig mit dem Text von
%	nummerierten Einträgen gleicher Ebene gesetzt.
% * nonumberline (Voreinstellung):
%	Die Eigenschaft numberline wird für das Inhaltsverzeichnis gelöscht.
%
%-------------------------------------------------------------------------------

% \KOMAOption{toc}{numberline}


%-------------------------------------------------------------------------------
% Option: tocdepth=[numberline|nonumberline]
%	Durch Setzen dieses Zählers kann bestimmt werden, bis zu welcher 
%	Gliederungsebene Einträge in das Inhaltsverzeichnis erfolgen sollen.
%	-1: part
%	 0: chapter
%	 ...
%	 
%-------------------------------------------------------------------------------

% \KOMAOption{tocdepth}{2}

%-------------------------------------------------------------------------------
% Generierung des Inhaltsverzeichnisses;
%	Zur korrekten Wiedergabe sind mindestens zwei Latex-Läufe notwendig.
%-------------------------------------------------------------------------------


\tableofcontents
