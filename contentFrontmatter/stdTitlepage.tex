%-------------------------------------------------------------------------------
%-------------------------------------------------------------------------------
% # Header für die Titelei
% 		Quelle: Dokumentation für KOMA-Script
%		Die Anmerkungen zu den einzelnen Optionen sind nicht vollständig,
% 		im Zweifelsfall muss die offizielle Dokumentation herangezogen werden. 
%
%-------------------------------------------------------------------------------
%-------------------------------------------------------------------------------


%-------------------------------------------------------------------------------
% Bei Standardklassen existiert maximal eine Titelseite mit den drei Angaben:
% * Title	("\title")
% * Autor ("\author")
% * Datum ("\date")
% 
% Bei den KOMAScript-Klassen können mit "\maketitle" bis zu 6 Titelseiten
% gesetzt werden. Außerdem  gibt es bei KOMAScript ein optionals nummerisches 
% Argument. Weiter Erklärung: KOMAScript-Dokumentation S.66.
% 
% \maketitle[Seitenzahl(optional)]
% 
% Das Setzen der Titelei erfolgt immer durch "\maketitle". Alle Anweisungen
% die den Inhalt der Titelei festlegt sind zwingend vor "\maketitle" zu verwenden.
% Es ist nicht empfehlenswert diese Anweisungen in der Dokumentenpräambel
% vor "\begin{document}" zu verwenden.
% 
% Anweisungen:
% * \extratitel{schmutztitel}
% 		Extraseite vor dem eigentlichen Haupttitel;
% 		enthält Verlagsangaben, Buchreihennummer, ...
% 	
% 		Bei KOMAScript ist es möglich auf dem Schmutztitel beliebigen Text, 
% 		auch mehrere Absätze, zu setzen.
% 	
% 		Beispiel:
% 			\documentclass{scrbook}
% 			\begin{document}
% 				\extratitle{Me}
% 				\title{It's me}
% 				\maketitle
% 				\end{document}
% * \titlehead{Kopf} (Blocksatz, kann frei gestaltet werden)
% * \subject{Typisierung}
% * \title{Titel}
% * \subtitle{Untertitel}
% * \author{Autor}
% * \date{Datum}
% * \publishers{Verlag}
% * \and
% * \thanks{Fußnote}
%
%	Anmerkungen:
%	* Um die Schriftstile der einzelnen Umgebungen zu ändern siehe Header
%		der einzelnen Koma-Klassen scrrprt, scrbook und scrartcl. 
%	* Bis auf "Kopf" und eventuelle Fußnoten werden alle Ausgaben hlrizontal 
%		zentriert. (weitere Beispiele KOMAScript-Dokumentation S. 69)
%	* Die Haupttitelseite ist nicht der Buchumschlag oder Buchdeckel!
%		Es handelt sich aber um die Titelseite im Buchblock.
%		Die Titelseite gehorcht den Randvorgaben für doppelseitige Satzspiegel.
%		Der Buchdeckel (das Cover) wird in einem getrennten Dokument erstellt.
%	* Allerdings kann die erste von "\maketitle" ausgegebene Titelseite
%		alternativ auch als Umschlagseite formatiert werden:
%			Option titlepage=firstiscover
%
%-------------------------------------------------------------------------------

\titlehead{Kopf}
\subject{Typisierung}
\title{Titel}
\subtitle{Untertitel}
\author{Autor}
\date{Datum}
\publishers{Verlag}
%\thanks{Fußnote} -- buggy


%-------------------------------------------------------------------------------
% Option: \uppertitleback{Titelrückseitenkopf}
% Rückseitenkopf der (standarmäßig leeren) Hauptitelseite im zweiseitigen Druck.
%
%-------------------------------------------------------------------------------

% \uppertitleback{Titelrückseitenkopf}

%-------------------------------------------------------------------------------
% Option: \lowertitleback{Widmung}
% Rückseitenfuß der (standarmäßig leeren) Hauptitelseite im zweiseitigen Druck.
%
%-------------------------------------------------------------------------------

% \lowertitleback{Titelrückseitenfuß}

%-------------------------------------------------------------------------------
% Option: \dedication{Widmung}
% KOMAScript bietet eine Widmungsseite.
%-------------------------------------------------------------------------------

%\dedication{Widmung}

%-------------------------------------------------------------------------------
% Generierung der Titelei: \maketitle immer zum Schluss
%-------------------------------------------------------------------------------

\maketitle

