%-------------------------------------------------------------------------------
%-------------------------------------------------------------------------------
% # Header für die KOMAScript-Klasse "scrartcl"
% 		Quelle: Dokumentation für KOMA-Script
%		Die Anmerkungen zu den einzelnen Optionen sind nicht vollständig,
% 		im Zweifelsfall muss die offizielle Dokumentation herangezogen werden. 
%
%-------------------------------------------------------------------------------
%-------------------------------------------------------------------------------

\documentclass{scrartcl}

%-------------------------------------------------------------------------------
% ## Frühe oder späte Optionenwahl
% 
% Diese besonderheit betrifft alle KOMA-Script Pakete und Klassen.
% 
% * "normaler" Latex-Weg Optionen festzulegen:
% 	\documentclass[Optionenliste(optional)]{KOMA-Script-Klasse}
% 	\usepackage[Optionenliste(optional)]{Paket-Klasse}
% 	Außer an die Klassen werden diese Optionen auch an alle Pakete 
% 	weitergereicht die diese verstehen.
% 
% 		Optionenliste := [ Option_1, ... ,Option_n]
% 
% * KOMA-Script eigene Alternative:
% 
% 	Bei Komascript haben die meisten Optionen einen Wert:
% 
% 		Optionenliste := [ Option_1=Value_1, ..., Option_n=Value_n]
% 
% 	Bei Verwendung einer KOMA-Script Klasse ist das Laden der Pakete
% 	"typearea" und "scrbase" überflüssig.
% 
% ### Wertzuweisungen von KOMA-Script Paketen:
% 
% Wertzuweisungen mit Latex-Längen und -Zählern sollten nie in Optionenlisten  
% der Befehle "\documentclass" und "\usepackage" geschehen, sondern nur über die  
% Befehle "\KOMAoptions{Optionenliste}" und "\KOMAoption{Option}{Werteliste}".
% So können Werte nach dem Laden einer Klasse geändert werden.
% Jede Option der Optionenliste hat die Form "Option=Wert". 
% Besonderheit:
% * Einige Optionen besitzten einen Säumniswert (engl. default value).
% * Einige Optionen können meherer Werte besitzen;
% 	diese können mehrfach durch Komma getrennt in der Optionenliste auftauchen.
% * Soll ein Wert ein Gleichheitszeichen oder ein komma enthalten,
% 	so ist der Wert in geschweifte Klammern zu setzen.
% 
% Standardwerte für alle einfachen Schalter in KOMA-Script
%
% 	Wert	Bedeutung
%	----    ---------
% 	true	aktiviert die Option
% 	on		"-"
% 	yes		"-"
% 	false	deaktiviert die Option
% 	off		"-"
% 	no		"-"
% 	
%	Unterschied zwischen voreingestelltem Wert und default values:
%	* Für alle Optionen des KOMAScript gibt es voreingestellte Werte.
%	  	Diese werden verwendet wenn eine Option nicht explizit genannt wird.
%	* Default values kommen nur bei Angabe einer Option ohne Wert zum Tragen.
%
%	Anmerkung zur Verwendung von Latex-Längen zur Berechnung von Werten:
%	Die Verwendung einer Latex-Länge wie "\baselineskip" zur Berechnung 
%	eines Wertes ist nicht deren Größe zum Zeitpunkt des Setzens der Option
%	sondern zum Zeitpunkt der Berechnung ausschlaggebend.
%	
%-------------------------------------------------------------------------------

%-------------------------------------------------------------------------------
% 	Beispiel zur Optionenwahl:
%		Standard-Klasse
% 			\documentclass[a4paper]{report}
%			\usepackage[BCOR=8.5mm, DIV=12]{typearea}
%		Koma-Klasse (das explitize Laden von "typearea" entfällt)
%			\documentclass[a4paper,BCOR=8.5mm]{scrrept}
%			\documentclass{scrrpt}
%			\KOMAoptions{BCOR=8.5mm, DIV=12}
%
%-------------------------------------------------------------------------------

%-------------------------------------------------------------------------------
% 	Option: draft = [true|false]
%
%	Schaltet den Entwursmodus ein oder aus.	
%	* voreingestellter Wert: false
%
%	Entwursmodus:
%	* überlange Zeilen werden mit schwarzen Kästchen markiert.
%
%-------------------------------------------------------------------------------

\KOMAoption{draft}{true}

%-------------------------------------------------------------------------------
% 	Option: fontsize = Größe [pt|em|...]
%
%	Legt die Größer der Grundschrift im Dokument fest.	
%	* voreingestellter Wert: 11pt (Standardklassen 10pt)
%	* Wird die Größe ohne Einheit angegeben wird pt als Einheit angenommen.
%	* Wird die Option innerhalb des Dokuments gesetzt, so werden ab diesem 
%	  	Punkt die Grundschriftgröße und alle davon abhängigen Größen geändert;
%		allerdings sollte die Option keinesfalls als Ersatz für "fontsize"
%		missverstanden werden!
%
%	Eine Änderung führt nicht zur automatischen Neuberechnung des 
%	Seitenspiegels.
%
%-------------------------------------------------------------------------------

\KOMAoption{fontsize}{11pt}

%-------------------------------------------------------------------------------
% 	Option: \setkomafont{Element}{Befehle}
%			Schriftumschaltung eines Elements wird mit einer völlig neuen
%			Definition versehen.
%
% 	Option: \addtokomafon{Element}{Befehle}
%			Existierende Definition der Schriftumschaltung wird erweitert.
%
%	Die Befehle "\setkomafont" und "\addtokomafont" sollten nur in der 
%	Dokumentenpräambel verwendet werden.
%
% 	Option: \usekomafont{Element}
%			die aktuelle Schriftart kann auf diejenige umgeschaltet werden,
%			die für das angegebene Element definiert ist.
%			
%			Beispiel:	
%			Angenommen für das Element "captionlabel" soll dieselbe Schriftart
%			woe für descriptionlaber verwendet werden.
%		
%			\setkomafont{captionlabel}{%
%				\usekomafont{descriptionlabel}%
%			}
%
%						
%	Tabelle: 
%	Als Argumente von "\setkomafont" und "\addtokomafont"  zulässige Befehle
%	(Erklärungen ergänzen)
%
%	\normalfont
%	\rmfamily
%	\sffamily
%	\ttfamily
%	\mdseries
%	\bfseries
%	\upshape
%	\itshape
%	\slshape
%	\scshape
%	\Huge
%	\huge
%	\LARGE
%	\Large
%	\large
%	\normalsize
%	\small
%	\footnotesize
%	\scriptsize
%	\tiny
%	\normalcolor
%
%
%	Tabelle: 
%	Elemente deren Schrift bei scrbook, scrreprt, scrartcl 
%	mit "\setkomafont" und "\addtokomafont" verändert werden können
%		
%	author
%		Autorenangaben im Haupttitel des Dokuments mit \maketitle,
%		also das Argument von "\author".
%	caption
%		text einer Abbildungs- oder Tabellenunter- oder -überschrift.
%	captionlabel
%		text einer Abbildungs- oder Tabellenunter- oder -überschrift;
%		Anwendung erfolgt nach dem Element "caption".
%	chapter
%		Überschrift der Ebene "\chapter".
%	chapterentry
%		Inhaltsverzeichniseintrag der Ebene "\chapter"
%	chapterentrypagenumber
%		Seitenzahl des Inhaltsverzeichnseintrages der Ebene "\chapter"
%		abweichend vom Element "\chapterentry".
%	chapterprefix
%		Kapitelnummernzeile bei Einstellungen "chapterprefix=true" oder
%		"appendixprefix=true".
%	date
%		Datum im Haupttitel des Dokuments mit \maketitle,
%		also das Argument von "\date".
%	dedication
%		Widmung nach dem Haupttitel des Dokuments mit "\maketitle";
%		also das Argument von "\dedication".
%	descriptionlabel
%		Label, also das optionale Argument der "\item"-Anweisung
%		in einer "description"-Umgebung.
%	dictum
%		mit "\dictum" gesetzter schlauer Spruch.
%	dictumauthor
%		Urheber des schlauen Spruchs;
%		Anwendung erfolgt nach dem Element "dictum".
%	dictumtext
%		Alternative bezeichnung vür "dictum".
%	disposition
%		Alle Gliederungsüberschriften, also die Argumente von "\part" bis
%		"\subgraph" und "\minisec" sowie die Überschrift der Zusammenfassung;
%		Anwendung erfolgt vor dem Element der jeweiligen Gliederungsebene.
%	footnote
%		Marke und Text einer Fußnote.
%	footnotelabel
%		Marke einer Fußnote;
%		Amwemdimg erfolgt nach dem Element "\footnote".
%	footnotereference
%		Referenzierung der Fußnotenmarke im Text.
%	footnoterule
%		Linie über dem Fußnotenapparat.
%	labelinglabel
%		Label, also das optionale Argument der "\item"-Anweisung,
%		und Trennzeichen, also das optionale Element der "labeling"- Umgebung
%		in einer "labeling"-Umgebung.
%	labelingseparator
%		Trennzeichen, also das optionale Argument der "labeling"-Umgebung,
%		in einer "labeling"-Umgebung;
%		Anwendung erfolgt nach dem Element "labelinglabel".
%	minisec
%		mit "\minisec" gesetzte Überschriften.
%	pagefoot
%		wird nur verwendet, wenn das Paket "scr-scrpage" oder "scrpage2"
%		geladen ist.
%	pagehead
%		alternative Bezeichung für "pageheadfoot".
%	pageheadfoot
%		Seitenkopf und Seitenfuß bei allen von KOMAScript definierten
%		Seitenstielen.
%	pagenumber
%		Seitenzahl im Kopf oder Fuß der Seite.
%	pagination
%		alternative Bezeichnung für "pagenumber". 
%	paragraph
%		Überschrift der Ebene "\paragraph".
%	part
%		Überschrift der Ebene "\part", jedoch ohne die Zeile 
%		mit der Nummer des Teils.
%	partentry
%		Inhaltsverzeichnis der Ebene "\part".
%	partentrypagenumber
%		Seitenzahl des Inhaltsverzeichniseintrags der Ebene "\part"
%	abweichend vom Element "partentry".
%	partnumber
%		Zeile mit der Nummer des Teils in Überschrift der Ebene "\part".
%	publishers
%		Verlagsangabe im Haupttitel des Dokuments mit "\maketitel",
%		also das Argument von "\publishers".
%	section
%		Überschrift der Ebene "\section".
%	sectionentry
%		Inhaltsverzeichniseintrag der Ebene "\section"; 
%		(nur scartcl).
%	sectionentrypagenumber
%		Seitenzahl des Inhaltsverzeichniseintrags der Ebene "\section"
%		abweichend vom Element "sectionentry"; 
%		(nur scartcl).
%	sectioning
%		alternative Bezeichnung für "disposition".
%	subject
%		Typisierung des Dokuments, also das Argument von "\subject"
%		auf der Haupttitelseite mit "\maketitle".
%	subparagraph
%		Überschrift der Ebene "\subparagraph".
%	subsection
%		Überschrift der Ebene "\subsection".
%	subsubsection
%		Überschrift der Ebene "\subsubsection".
%	subtitle
%		Untertitel des Dokuments,
%		also das Argument von "\subtitle" auf der Haupttitelseite mit 
%		"\maketitel".
%	titlehead
%		Kopf über dem Haupttitel des Dokuments, 
%		also das Argument von "\titlehead" auf der Haupttitelseite mit 
%		"\maketitel". 
%
%-------------------------------------------------------------------------------


%-------------------------------------------------------------------------------
% 	Option: titleapge = [true|false|firstincover]
%
%	Die Anweisung "\maketitle" verwendet "titlepage"-Umgebungen zum Setzen 
%	dieser Seiten, die normalerweise weder Seitenkopf noch Seitenfuß enthalten.
%
%	* "true":
%		Mit dieser Option wird ausgewählt, ob dür die mit "\maketitle" 
%		gesetzte Titelei eigene Seiten verwendet werden, oder statdessen die 
%		Titelei von "\maketitle" als Titelkopf gesetz wird.
%
%	* "false":
%		Ist diese Option deaktiviert wird die Titelei lediglich speizell 
%		hervorgehoben, auf der Seite mit dem Titel kann aber weiteres Material
%		gesetz werden.
%
%	* "firstincover":
%		Titelseiten werden aktiviert und die erste von "\maketitel" ausgegeben 
%		Titelseite als (Schmutztitel oder Haupttitel) wird als Umschlagseite
%		ausgegeben.		
%
%		Die Ränder dieser Umschlagseite werden über 
%		*	"\coverpagetopmargin" 		(oberer Rand)
%		*	"\coverpageleftmargin" 		(linker Rand)
%		*	"\coverpagerightmargin" 	(rechter Rand)
%		*	"\coverpagebottommargin" 	(rechter Rand)
%	
%			Die Voreinstellungen sind von den Längen "\topmargin" und 
%			"\evensidemargin" abhängig und können mit renewcomand geändert werden.
%	
%	* voreingestellter Wert: false
%	* default value: "true"
%
%	Anmerkungen: 
%	* Bei KOMAScript wurde die Titelei gegenüber 
%	* Alternativ zu den Standardtiteleien der Klasse kann eine eigene Titelei
%		mit der "titlepage" Umgebung gestaltet werden. 
%
%-------------------------------------------------------------------------------

\KOMAoption{titlepage}{true}


%-------------------------------------------------------------------------------
%  	Option: abstract = [true|false]
%
%	* voreingestellter Wert: false
%	* default value: "true"
%
%	Bei Artikeln oder Berichten findet man unmittelbar unter der Titelei und 
%	vor dem Inhaltsverzeichnis eine Zusammenfassung.
%	Bei Verwendung von Titelseiten wird die Zusammenfassung eher als Kapitel
%	oder Abschnitt gesetzt, nicht für "scrbook" (siehe "\chapter*"). 
%	Diese Option legt fest, ob die Zusammenfassung eine Überschrift bekommt oder 
%	nicht aufgenommen wird oder nicht.
%	
%-------------------------------------------------------------------------------

% \KOMAoption{abstract}{true}


%-------------------------------------------------------------------------------
% 	# Absatzauszeichnung 
%
% 	Option: parskip=Methode 
%	Diese Option legt die Absatzauszeichnung fest.	
%	* voreingestellter Wert: false
%	* default value: "true"
%
%	Varianten der Absatzauszeichung:
%	* Absatzeinzug:
%		+ bessere Absatzauszeichung als nur
%		- kann anstrengend zu lesen sein
%	* Absatzabstand:
%		+ deutliche optische Trennung
%		- kann in verschiedenem Zusammenhang leicht verloren gehen.
%			(z.B. nach Formel, Seitenanfang, ...)
%	
%	Anmerkung:
%	* Bei Verwendung eines normalen Satzspiegels ist eine Absatzauszeichung  
%		durch Absatzeinzug und ohne Absatzabstand die vorteilhafteste Wahl 
%		(Die Standardklassen setzen so).
%	* Eine Kombination von Absatzeinzug und Absatzabstand ist wegen 
%		übertriebener Redundanz nicht zu empfehlen.
%
%	Argumente:
%	* [false | off | no]:
%		Absätze werden durch einen Einzug der ersten Zeilen von einem 
%		Geviert (1em) gekennzeichnet.
%		Der erste Absatz eines Abschnitts wird nicht eingezogen.
%	* Methode
%	* never:
%		Es wird kein Abstand zwischen Absätzen eingefügt, wenn für den 
%		vertikalen Ausgleich der Einstellung "\flushbottom" zusätzlich 
%		vertikaler Astand verteilt werden muss.
%	
%	Methode: Arg1Arg2
%	Die Methode setzt sich aus zwei Teilen zusammen. 
%	* Arg1:
%		* [full | true | on | yes] (Methode Teil 1):
%			Absätze werden durch einen vertikalen Abstand von einer Zeile 
%			gekennzeichnet.
%		* half (Methode Argument 1):
%			Absätze werden durch einen vertikalen Abstand von einer Zeile 
%			gekennzeichnet.
%	* Arg2 (kann auch entfallen):
%		"*": 		Ein Viertel der normalen Zeilenlänge wird freigelassen.
%		"+": 		Ein Drittel der normalen Zeilenlänge wird freigelassen.
%		"-": 		keine Vorkehrung
%		default:	Lässt man das Zeichen weg wird in der letzten Zeile des 
%					Absatzes mindestens ein Geviert (1 em) freigelassen.
%
%	Anmerkungen:
%	* Wird ein Absatzabstand verwendet, so verändert sich auch der Abstanf vor,
%		nach und innerhalb von Listenumgebungen. Dadurch wird verhindert, 
%		dass diese Umgebbungen oder Absätze innerhalb dieser Umgebungen
%		stärker vom Text abgesetzt werden als die Absätze des normalen
%		Texts voneinander.
%	* Die Einstellung kann jederzeit geändert werden.
%
%-------------------------------------------------------------------------------

% \KOMAoption{parskip}{true}
		
%-------------------------------------------------------------------------------
%-------------------------------------------------------------------------------
% 	# Kopf und Fuß bei vordefinierten Seitenstilen
%
%-------------------------------------------------------------------------------

%-------------------------------------------------------------------------------
%	Option: headsepline=[true|false]
%			footsepline=[true|false]
%
%	* voreingestellter Wert: false
%	* default value: "true"
%
%	Mit dieser Option kann eingestellt werden ob unter dem Kolumnentitel
%	oder über dem Fuß eine horizontale Linie gewünscht wird.	
%
%	Anmerkungen:
%	* Bei den Seitenstielen "empty" und "plain" hat die Option "headsepline"
%		keine Auswirkungen, da hier auf einen Seitenkopf verzichtet werden soll.
%	* Eine Linie hat Auswirkungen auf die Berechnung des Satzspiegels:
%		* bei "headsepline" wird auch "headinclude" aktiviert.
%		* bei "footsepline" wird auch "footinclude" aktiviert.
%
%-------------------------------------------------------------------------------


% \KOMAoption{headsepline}{true}
% \KOMAoption{footsepline}{true}
	
%-------------------------------------------------------------------------------
%	Option: \pagestyle{Seitenstil}
%		 	\thispagestyle{Seitenstil}
%
%	Üblicherweise wird zwischen vier Seitenstielen unterschieden:
%	* empty:
%		Kopf- und Fußzeile bleiben leer, identisch zu den Standardklassen.
%	* headings:
%		Seitenstiel für lebende Kolumnentitel (engl. running headline).
%	* myheadings:
%		Entspricht weitgehend dem Seitenstiel "headings, allerdings werden
%		Kolumnentitel nicht automatisch erzeugt, sondern liegen in der 
%		Verantwortung des Autors.
%	* plain:
%		Keinerlei Kolumnentitel, nur Seitenzahl wird ausgegeben.
%		* einseitiger Satz: mittig im Fuß
%		* doppelseitiger Satz: außen im Fuß
%		
%	Anmerkung:
%	* Der Seitenstil kann jederzeit mit "\pagestyle" gesetzt werden.
%	* Der Seitenstil eienr einzelnen Seite kann mit "\thispagestyle" 
%		gesetzt werden.
%		
%	Tabelle:	
%	Ausgezeichnete Seitenstile
%	* "\titlepagestyle"
%		Seitenstil der Titelei
%	* "\partpagestyle"
%		Seitenstil der Seiten mit "part"-Titeln
%	* "\chapterpagestyle"
%		Seitenstil der kapitelanfangseiten
%	* "\indexpagestyle"
%		Seitenstil der ersten Stichwortverzeichnisseite
%
%-------------------------------------------------------------------------------

% \KOMAoption{pagestyle}{plain}

%-------------------------------------------------------------------------------
%	Option: \pagenumbering{Nummerierungsstil}
%	
%	Mit dieser Option kann de Nummerierungsstil angepasse werden. 
%	
%	Nummerierungsstile:
%	* arabic:	arabische zahlen
%	* roman:	kleine römische Zahlen
%	* Roman: 	große römische Zahlen
%	* alph:		Kleinbuchstaben
%	* Alph:		Großbuchstaben
%
%	Anmerkungen:
%	* Die Umschaltung gilt ab der Seite auf der die Anweisung aufgerugen wird.
%	* Der Aufruf "\pagenumbering" setzt immer die Seitenanzahl zurück -
%		die aktuelle Seite bekommt Nummer 1 des aktuellen Nummerierungsstils.
%
%-------------------------------------------------------------------------------

% \pagenumbering{arabic}	

%-------------------------------------------------------------------------------
%	# Vakatseiten
%
%	Option: \cleardoublepage{Seitenstil}
%
%	Vakatseiten sind Seiten, die beim Satz eines Dokuments absichtlich leer 
%	bleiben. Bei Latex werden sie jedoch in der Voreinstellung mit dem aktuell
%	gültigen Seitenstil gesetzt; KOMAScript bietet diverse Erweiterungen.
%
%	Vakatseiten findet man hauptsächlich im doppelseitigen Satz,
%	da Kaptiel immer auf der rechten Seite beginnen aber auch auf einer 
%	rechten Seite enden können.
%	
%	Optionen S. 85 KOMAScript-Dokumentation
%
%-------------------------------------------------------------------------------

% \cleardoublepage{plain}

%-------------------------------------------------------------------------------
%	# Fußnoten
%
%	Option: footnotes=Einstellung
%	* voreingestellter Wert: nomultiple
%	* default value: multiple
%
%	
%	Einstellung:
%		* multiple:
%			Unmittelbar aufeinanderfolgende Fußnotenmarkierungen weren durch
%			"\multfootsep" voneinander getrennt ausgegeben.
%		* nomultiple:
%			Unmittelbar aufeinander folgende Fußnotenmerkierungen werde auch 
%			unmittelbar aufeinander folgend ausgegeben.
%
%	Anmerkungen:
%	* Es ist jederzeit möglich die Einstellungen für die Fußnoten umzustellen
%	* Weitere Einstellungsmöglichkeiten für Fußnoten ab 
%		S.87 KOMAScript-Dokumentation
%
%-------------------------------------------------------------------------------

\KOMAoption{footnotes}{multiple}


%-------------------------------------------------------------------------------
%	# Abgrenzungen (Vorspann, Hauptteil, Nachspann)
%
%	Bei Büchern gibt es teilweise die Grobaufteilung in Vorspann, Hauptteil
%	und Nachspann.
%
%	Entsprechungen in KOMASCript:	
%	* \frontmatter:
%		Vorspann, römische Nummerierung; 
%		das Vorwort kann als normales Kapitel gesetzt werden (Kapitelnummer 0).
%		(addsec und section* verwenden).
%	* \mainmatter:
%		Hauptteil, arabische Seitenzahlen, Seitenzälung beginnt neu mit 1;
%		existiert kein Vorspann kann diese Anweisung entfallen.		
%	* \backmatter:
%		Nachspan, (standerdmäßig) keine getrennte Seitennummerierung -
%		sollte diese nötig sein "pagenumbering verwenden";
%		Was in den Anhang kommt ist unterschiedlich:
%		manchmal nur Literaturverzeichnis, manchmal nur Index, 
%		manchmal der komplette Anhang.
%
%
%
%-------------------------------------------------------------------------------


%-------------------------------------------------------------------------------
%	# Gliederung
%
%	Die Standardgliederungsbefehle funktionieren bei den KOMAScript-Klassen wie
%	bei den Standardklassen.
%	In der Voreinstellung kan über ein optionales Argument ein abweichender Text
%	für den Kolumnentitel und das Inhaltsverzeichnis vorgegeben werden.
%	
%	Tabelle:
%	Gliederungsbefehle:
%	* "\part[Kurzform]{Überschrift}"			nur scrbook, scrrprt
%	* "\chapter[Kurzform]{Überschrift}"			nur scrbook, scrrprt
%	* "\section[Kurzform]{Überschrift}"
%	* "\subsection[Kurzform]{Überschrift}"
%	* "\subsubsection[Kurzform]{Überschrift}"
%	* "\paragraph[Kurzform]{Überschrift}"
%	* "\subparagraph[Kurzform]{Überschrift}"
%
%	Anmerkungen:
%	* Es können unterschiedliche Kurzformen für das Inhaltsverzeichnis und 
%		den Kolumnentitel gewählt werden:
%		
%			\gliederungsbefehl[head=Kurzform1,tocentry=Kurzform2]{Überschrift}
%	 
%		Um ein Gleichheitszeichen oder ein Komma in einem dieser beien Werte 
%		unterzubringen muss dieser in geschweifte klammern gesetzt werden.
%	* Sternvarianten:
%		Bei den Sternvarianten der Gliederungsbefehle erfolgt keine nummerierung,
%		wird kein Kolumnentitel gesetzt und kein Eintrag im Inhaltsverzeichnis 
%		vorgenommen:
%
%			\gliederungsbefehl*{Überschrift}
%		
%		Der Verzicht auf den Kolumnentitel hat oft den unerwünschten Effekt,
%		das der Titel des letzten kapitels wieder auftaucht. 
%		Lösung: add<gliederungsbefehl>
%	* add<Gliederungsbefehl>[Kurzform]{Überschrift}, 
%	  add<Gliederungsbefehl>*{Überschrift}
%	  mit <Gliederungsbefehl> aus [part,chapter,sec]: 
%		Diese Gliederungsbefehle gleichen den Gliederungsbefehlen der 
%		Standardklassen, mit dem Unterschied, dass die Kolumnentitel 
%		gelöscht werden - der Kolumnentitel bleibt leer.
% 
%	SemiGliederungsbefehl: "\minisec{Überschrift}"
%		Hervorgehobene Überschrift die eng mit dem nachfolgenden Text zusammenhängt.
%		Diese Überschrift ist keiner Gliederungsebene zugeordnet;
%		sie wird nicht im Inhaltsverzeichnis aufgebommen und erhält keine 
%		Nummerierung.
%
%-------------------------------------------------------------------------------

%-------------------------------------------------------------------------------
%	Option: headings = Einstellungen
%
%	Einstellung der Überschriftengröße. Eine Besonderheit von KOMAScript 
%	betrifft die Behandlung des optionalen Arguments der Gliederungsbefehle.
%	Sowohl dessen Funktionals auch dessen Bedeutung kann durch die 
%	Einstellungen "headings=optiontohead", "headings=optiontotoc" und 
%	"headings=optionstoheadandtoc" beeinflusst werden.
%	
%	Einstellungen
%	* big:
%		Große Überschriften mit großen Abständen darüber und darunter
%	* normal:
%		Mittelgroße Überschriften mit mittelgroßen Abständen darüber und darunter.
%	* small: 
%		Kleine Überschriften mit kleine Abständen darüber und darunter.
%	* onlinechapter:
%		Kapitelüberschriften werden wie andere Überschriften auch gesetzt.
%	* twolinechapter:
%		Kapitelüberschriften werden mit einer Vorsatzzeile gesetztm
%		deren Inhalt von "\chapterformat" bestimmt wird.
%	* onelineappendix:
%		Kapitelüberschriften im Anhang werden wie andere Überschriften auch
%		gesetzt.
%	* twolineappendix:
%		Kapitelüberschriften im Anhang werden mit einer Vorsatzzeile gesetztm
%		deren Inhalt von "\chapterformat" bestimmt wird.
%	* optiontohead:
%		Die erweiterte Funktion des optionalen Arguments der Gliederungsbefehle
%		wird aktiviert. In der Voreinstellung wird das optionale Argument 
%		ausschließlich für den Kolumnentitel verwendet.
%	* optiontoheadandtoc:
%		Die erweiterte Funktion des optionalen Arguments der Gliederungsbefehle
%		wird aktiviert.	In der Voreinstellung wird das optionale Argument 
%		für den Kolumnentitel und die Eintragung ins Inhaltsverzeichnis 
%		verwendet.
%	* optiontotoc:
%		Die erweiterte Funktion des optionalen Arguments der Gliederungsbefehle
%		wird aktiviert. In der Voreinstellung wird das optionale Argument 
%		ausschließlich für die Eintragung ins Inhaltsverzeichnis verwendet.
%			
%	Anmerkungen:
%	* Die Abstände vor und nach den Kapitelüberschriften werden von dieser
%		Einstellung ebenfalls beeinflusst.
%
%-------------------------------------------------------------------------------

% \KOMAoption{headings}{small}

%-------------------------------------------------------------------------------
%	Option: numbers = Einstellungen
%	
%	nach Duden:
%	* Gliederungen mit ausschließlich arabischen Nummern:
%		kein abschließender Punkt
%	* Gliederungen sonst:
%		abschließender Punkt.
%		
%	Einstellungen
%	* autoendperiod:
%		Automatismus
%	* endperiod:
%		Gliederungen werden mit Punkt beendet.
%	* noendperiond:
%		Gliederungen werden ohne Punkt beendet.
%
%	Anmerkungen:
%	* Diese Option kann nur in der Präambel eingestellt werden.
%
%-------------------------------------------------------------------------------

% \KOMAoption{numbers}{autoendperiod}

%-------------------------------------------------------------------------------
%	Option: chapterlist = Wert
%
%	Normalerweise führt jeder mit "\chapter" und "\addchap" erzeugte  
%	Kapiteleintrag einen vertikalen Abstand in die Verzeichnisse der 
%	Gleitumgebung ein (siehe auch "listof").
%	Mit der Option "chapterlists" kann der Abstand verändert werden.
%
%	Wert:
%	* chaptergapsmall: 
%		10pt
%	* entry: 
%		Statt des Abstandes wird der Kapiteleintrag in die Verzeichnisse
%		aufgenommen. Dieser Eintrag erfolgt auch, wenn in dem Kapitel 
%		keine Gleitumgebungen vorkommen.
%
%-------------------------------------------------------------------------------

% \KOMAoption{chapterlist}{entry}
