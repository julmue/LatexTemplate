%-------------------------------------------------------------------------------
%-------------------------------------------------------------------------------
% # Allgemeiner Header für KOMAScript-Dateien mit kurzer Dokumentation
% 		Quelle: Dokumentation für KOMA-Script
%		Die hier aufgeführten Optionen sind für alle KOMAScript Klassen gültig.
%		Die Anmerkungen zu den einzelnen Optionen sind nicht vollständig,
% 		im Zweifelsfall muss die offizielle Dokumentation herangezogen werden. 
%
%-------------------------------------------------------------------------------
%-------------------------------------------------------------------------------


%-------------------------------------------------------------------------------
% ## Frühe oder späte Optionenwahl
% 
% Diese besonderheit betrifft alle KOMA-Script Pakete und Klassen.
% 
% * "normaler" Latex-Weg Optionen festzulegen:
% 	\documentclass[Optionenliste(optional)]{KOMA-Script-Klasse}
% 	\usepackage[Optionenliste(optional)]{Paket-Klasse}
% 	Außer an die Klassen werden diese Optionen auch an alle Pakete 
% 	weitergereicht die diese verstehen.
% 
% 		Optionenliste := [ Option_1, ... ,Option_n]
% 
% * KOMA-Script eigene Alternative:
% 
% 	Bei Komascript haben die meisten Optionen einen Wert:
% 
% 		Optionenliste := [ Option_1=Value_1, ..., Option_n=Value_n]
% 
% 	Bei Verwendung einer KOMA-Script Klasse ist das Laden der Pakete
% 	"typearea" und "scrbase" überflüssig.
% 
% ### Wertzuweisungen von KOMA-Script Paketen:
% 
% Wertzuweisungen mit Latex-Längen und -Zählern sollten nie in Optionenlisten  
% der Befehle "\documentclass" und "\usepackage" geschehen, sondern nur über die  
% Befehle "\KOMAoptions{Optionenliste}" und "\KOMAoption{Option}{Werteliste}".
% So können Werte nach dem Laden einer Klasse geändert werden.
% Jede Option der Optionenliste hat die Form "Option=Wert". 
% Besonderheit:
% * Einige Optionen besitzten einen Säumniswert (engl. default value).
% * Einige Optionen können meherer Werte besitzen;
% 	diese können mehrfach durch Komma getrennt in der Optionenliste auftauchen.
% * Soll ein Wert ein Gleichheitszeichen oder ein komma enthalten,
% 	so ist der Wert in geschweifte Klammern zu setzen.
% 
% Standardwerte für alle einfachen Schalter in KOMA-Script
%
% 	Wert	Bedeutung
%	----    ---------
% 	true	aktiviert die Option
% 	on		"-"
% 	yes		"-"
% 	false	deaktiviert die Option
% 	off		"-"
% 	no		"-"
% 	
%	Unterschied zwischen voreingestelltem Wert und default values:
%	* Für alle Optionen des KOMAScript gibt es voreingestellte Werte.
%	  	Diese werden verwendet wenn eine Option nicht explizit genannt wird.
%	* Default values kommen nur bei Angabe einer Option ohne Wert zum Tragen.
%
%	Anmerkung zur Verwendung von Latex-Längen zur Berechnung von Werten:
%	Die Verwendung einer Latex-Länge wie "\baselineskip" zur Berechnung 
%	eines Wertes ist nicht deren Größe zum Zeitpunkt des Setzens der Option
%	sondern zum Zeitpunkt der Berechnung ausschlaggebend.
%	
%-------------------------------------------------------------------------------

%-------------------------------------------------------------------------------
% ## Satzspiegelberechnung mit dem Paket typearea.sty
% 
% ### Grundlagen der Satzspiegelberechnung
% (Alternative: geometry.sty)
% "typearea.sty" berechnet den Satzspiegel anhand bestimmter typografischer 
% Konstruktionsalgorithmen. Das gewünschte Ergebnis ist, dass der 
% Satzspiegel (der Textkörper) dem Seitenverhältnis der Seite entspricht.
% 
% 1)	 Satzspiegelhöhe: satzspiegelbreite = Seitenhöhe: Seitenbreite
% 2)		      oberer Rand: unterer Rand = 1 : 2
% 3)		     linker Rand : rechter Rand = 1 : 1 (nur bei einseitigem Dokument)
% 4)	  innerer Randanteil : äußerer Rand = 1:2 (nur bei beidseitigem Dokument)
% 5)						   Seitenbreite = Papierbreite - Bindekorrektur
% 6)		     oberer Rand + unterer Rand = Seitenhöhe - Satzspiegelhöhe
% 7)		 	 linker Rand + rechter Rand = Seitenbreite - Satzspiegelbreite
% 8)	  innerer Randanteil + äußerer Rand = Seitenbreite - Satzspiegelbreite
% 9)	innerer Randanteil + Bindekorrektur = Bundsteg
% 
% Breite des Satzspiegels abhängig von:
% 	* Größe, Laufweite und Art der verwendeten Schrift
% 	* Durchschuss (Abstand der Textzeilen) 
% 	  Bei größerem Durchschuss dürfen die Zeilen länger sein.
% 	  Optimum ca. 60 - 70 Zeichen einschließlich Leer- und Satzzeichen.
% 		-> anpassen mit Befehl "\linespread"; (besser) Paket "setspace" 
% 			* Durchschuss darf nicht zu klein sein:
% 				verfolgung der Zeile schwierig
% 			* Durchschuss darf nicht zu groß sein:
% 				Graueffekt der Seite gestört, wahrnehmbare Streifenbildung
% 	* Länge der Worte
% 	* verfügbarer Platz
% 	
% Für die Satzspiegelberechnung wichtige Eigenschaften von Latex:
% * Vertikaler Beginn der ersten Zeile:
% 	Latex beginnt die erste Zeile des Textbereichs einer Seite nicht am oberen 
% 	Rand des Textbereichs, sondern setzt die Grundlinie der Zeile mit einem 
% 	definierten Mindestabstand zum oberen Rand des Textbereichs.
% * Vertikaler Abschluss der letzten Zeile:
% 	* Befehl: "\raggedbottom"
% 		Vertikaler Punkt der letzten Zeile einer Seite liegt nicht auf einem 
% 		vorberstimmten Platz, sondern ist frei. 
% 		Bei beidseitigem DrucK nicht erwünscht.
% 	 * Befehl: "\flushbottom"
% 		Letzte Zeile einer Seite liegt immer auf dem unteren Rand des 
% 		Textbereichs. Dafür müssen geg. dehnbare Abstände (Absatzabstand,..) 
% 		über das Maß gestreckt werden.
% 		Um nicht unnötig eine Dehnung zu erzwingen sollte die Höhe des 
% 		Textbereichs ein Vielfaches der Textzeilenhöhe zuzüglich des Abstandes
% 		der ersten zeile vom oberen Rand des Textbereichs sein.
% 
% 
% ### Satzspiegelkonstuktion durch Teilung
% 
% Berechnung des Textbereiches der die Bedingungen 1-9 erfüllt:
% 1) Abzug BCOR (Bindekorrektur) von der Innenseite des Papiers.
% 2) Vertikale Teilung der restlichen Seite in eine Anzahl DIV 
% 	gleich breiter Streifen
% 3) Horizontale Teilung der restlichen Seite in eine Anzahl DIV 
% 	gleich breiter Streifen.
% 4) oberer Rand = der oberste Streifen; 
% 	unterer Rand = die untersten zwei Streifen
% 5) einseitiger Druck:
% 	rechter Rand = rechtester Streifen; linker Rand = linkester Streifen
%    zweiseitiger Druck:
% 	einseitiger Druck + 
% 	innerer Rand = innerster vertikaler Streifen
% 6) innerer Rand = innerer Rand + BCOR
% 
% Die Breite bzw. Höhe der Ränder des Textbereichs resultieren damit aus der 
% Anzahl DIV (DIV >= 4) der Streifen.
% 
% 
% ### Satzspieglekonstruktion durch Kreisschlagen
% 
% Bei diesem Verfahren will man die gleichen Werte nicht nur in Form des 
% Seitenverhältnisses wiederfinden; man geht außerdem davon aus, dass das 
% Optimum dann erreicht wird, wenn die Höhe des Textbereichs der Breiter der 
% Seite entspricht.
% Nachteil: die Breite des Textbereichs hängt nicht mehr von der Schriftart ab,
% der Setzer muss due zum Textbereich passende Schrift wählen.
% Beim "typearea"-Paket wird durch Auswahl eines ausgezeichneten DIV-Wertes 
% oder einer speziellen Paketoption der DIV-Wert ermittel, bei dem der 
% resultierende Satzspiegel dem Ideal der Satzspiegelkonstuktion durch
% Kreisschlagen am nächsten kommt.
% 
%-------------------------------------------------------------------------------

%-------------------------------------------------------------------------------
% 	Beispiel zur Optionenwahl:
%		Standard-Klasse
% 			\documentclass[a4paper]{report}
%			\usepackage[BCOR=8.5mm, DIV=12]{typearea}
%		Koma-Klasse (das explitize Laden von "typearea" entfällt)
%			\documentclass[a4paper,BCOR=8.5mm]{scrrept}
%			\documentclass{scrrpt}
%			\KOMAoptions{BCOR=8.5mm, DIV=12}
%
%-------------------------------------------------------------------------------

%-------------------------------------------------------------------------------
% Option: BCOR = Korrektur[mm]
%	* voreingestellter Wert: 0mm
%
%-------------------------------------------------------------------------------

\KOMAoption{BCOR}{0mm}

%-------------------------------------------------------------------------------
% Option: DIV = Faktor (Faktor >= 4)
%	DIV-Voreinstellungen für A4 bei Verzicht auf BCOR:	
%		Grundschriftgröße:	10pt	11pt	12pt
%		DIV:				 8		10		12
%	Achtung:
%	* Die Grundeinstellungen können bei kleinen Schriftarten zu unangenehm 
%		hohen Laufweiten führen - hier DIV berechnen lassen.
%	* Bei anderen Papierformaten DIV berechnen lassen.	
% 
% Mögliche symbolische Werte für die Option DIV 
% * areaset:
% 		Satzspiegel neu anordnen
% * calc:
% 		DIV-Berechnung durch Satzspgkonstr. durch Teilung
% * classic:
% 		DIV-Berechnung durch Satzspgkonstr. durch Kreisschlagen
% * current:
% 		Satzspgkonstr. mit dem aktuelle gültigen DIV-Wert erneut durchführen
% 		(nach Änderungen von ---)
% * default:
% 		Satzspgkonstr. mit dem Standardwert für das aktuelle Seitenformat und 
% 		die aktuelle Schriftgröße erneut durchführen;
% 		falls kein Standardwert existiert default andwenden.
% * last:
% 		Satzspiegelberechnung mit demselben DIV-Argument, das beim letzen
% 		Aufruf angegeben wurde, erneut durchführen.
%
% Anmerkung:
% *	Nach Änderung der Schriftart, Durchschuss o.ä. (auch fontenc) muss die  
% 	Berechnung erneut durchgeführt werden.
%	Beispiel:
%		\documentclass[10pt, twoside, BCOR=12mm, DIV=calc]{scrreprt}
%		\linespread{1.25}	oder \usepackage[onehalfspacing]{setspace}
%		\KOMAoptions{DIV=last}
% * Der Befehl \Typearea ist so definiert, dass es auch möglich ist,
%		mitten in einem Dokument den Satzspiegel zu wechseln;
%		dies führt allerdings zu einem Seitenumbruch.
%
%-------------------------------------------------------------------------------

\KOMAoption{DIV}{calc}

%-------------------------------------------------------------------------------
% Option: twoside=[true|false|semi]
% 	Ein oder doppelseitiger Satz des Dokuments.
%	* voreingestellter Wert: einseitig
% 	* default: "true"
% 	* semi führt zu doppelseitigem Satz mit einseitigen Rändern und einseitigen
%		Marginalien.
% 	* Eine Verwendung dieser Option nach dem Laden von typearea führt automatisch 	
%		zu einer Neuberechnung des Satzspiegels mittels \rectaltypearea.
%	* Verwendung von doppelseitigem Satz führt automatisch zur Verwendung 
%		von "\flushbottom".
%	* Verwendung von einseitigem Satz führt automatisch zur Verwendung 
%		von "\raggedbottom".
%	
%-------------------------------------------------------------------------------

\KOMAoption{twoside}{false}

%-------------------------------------------------------------------------------
% Option: twocolumn=[true|false]
%	Ein oder doppelspaltiger Satz des Dokuments.
%	* voreingestellter Wert: einspaltig
% 	* default: "true"
%
%-------------------------------------------------------------------------------

\KOMAoption{twocolumn}{false}

%-------------------------------------------------------------------------------
% Option: headinclude=[true|false]
%	Beim oberen und unteren Rand stellt sich die Frage wie Kopf- und 
%	Fußzeile zu behandeln sind. Gehören diese zum Textkörper oder zum 
%	jeweiligen Rand? Entscheiden ist, wie eine wohlgefüllte Seite bei 
%	unscharfer Betrachtung wirkt.
%	* leerer Kopf und leerer Fuß werden immer zum Rand gerechnet.
%
%	Oberer Rand:
%   * voreingestellter Wert: meist "false" (abhängig von der Klasse und Paketen)
%	* default value: "true"
%  	* Anhaltspunkte für den oberen Rand:
%	 	In der Kopfzeile wird häufig der Kolumnentitel gesetzt 
% 	 	(Kolumnentitel: wiederholung einer Überschrift mit Titelcharakter)
%	 	* lebende Kolumnentitel (+ Paginierung im Kopf):
%			eher zum Text gehörig.
%		 * schwierig: 
%			Paginierung im Fuß und sehr stark schwankende Länge der 
%			Kolumnentitel. Im Zweifelsfall Kopf zum Textkörper.
%		 * Bei Abtrennung des Kopfes durch eine geschlossene Linie:
%			unbedint zum Textkörper rechnen.
%		* default value = true
%
%	Eine Änderung führt nicht zur automatischen Neuberechnung des 
%	Seitenspiegels.
%
%-------------------------------------------------------------------------------

% \KOMAoption{headinclude}{false}

%-------------------------------------------------------------------------------
% Option: headinclude=[true|false]
%	Beim oberen und unteren Rand stellt sich die Frage wie Kopf- und 
%	Fußzeile zu behandeln sind. Gehören diese zum Textkörper oder zum 
%	jeweiligen Rand? 
%
%	Unterer Rand:
%   * voreingestellter Wert: meist "false" (abhängig von der Klasse und Paketen)
%	* default value: "true"
%	* Anhaltspunkte für den unteren Rand:
%		* Fuß der nur die Paginierung enthält sollte zum Rand gerechnet werden.
%
%
%	Eine Änderung führt nicht zur automatischen Neuberechnung des 
%	Seitenspiegels.
%
%-------------------------------------------------------------------------------

% \KOMAoption{footinclude}{false}

%-------------------------------------------------------------------------------
% Option: mpinclude=[true|false]
%	Option die festlegt ob Randnotizen zum Text gehören oder nicht.
%	Eine Breiteneinheit wird vom Text weggenommen und für Randnotizen verwendet. 
%	* voreingestellter Wert: "false"
%	* default value: "true"
%
%	Eine Änderung führt nicht zur automatischen Neuberechnung des 
%	Seitenspiegels.
%
%-------------------------------------------------------------------------------

% \KOMAoption{mpinclude}{false}

%-------------------------------------------------------------------------------
% Option .1: headlines=Zeilenanzahl
%	Anzahl der Zeilen in der Kopfzeile.
%	* voreingestellter Wert: 1.25 (meist gut für unterstrichene 
%										und nicht unterstrichene Kopfzeilen)
%	* default value: "true"
%
% Option .1: headheight=
%	* voreingestellter Wert:
%	* default value:
%
%	Eine Änderung führt nicht zur automatischen Neuberechnung des 
%	Seitenspiegels.
%
%-------------------------------------------------------------------------------

% \KOMAoption{headlines}{1.25}

%-------------------------------------------------------------------------------
% Option .1: footlines=Zeilenanzahl
%	Anzahl der Zeilen in der Kopfzeile.
%	* voreingestellter Wert: 1.25 (meist gut für unterstrichene 
%										und nicht unterstrichene Fußzeilen)
%	* default value: "true"
%
% Option .2: foodheight=
%	* voreingestellter Wert:
%	* default value:
%
%	Eine Änderung führt nicht zur automatischen Neuberechnung des 
%	Seitenspiegels.
%
%-------------------------------------------------------------------------------

% \KOMAoption{foodlines}{1.25}

%-------------------------------------------------------------------------------
% Weitere hier undokumentierte Optionen:
%	Befehl "\areaset"
%-------------------------------------------------------------------------------

%-------------------------------------------------------------------------------
% Option .1: paper=format [letter|legal|executive|an|bn|cn|dn] mit n aus Nat.
%	Einstellung des Papierformats
%	* voreingestellter Wert: a4
%	* default value: a4
%
% Option .2: paper=format [landscape|seascape|portrait] mit n aus Nat.
%	Einstellung des Papierformats
%	* voreingestellter Wert: portrait
%	* default value: lanscape
%
%	Eine Änderung führt nicht zur automatischen Neuberechnung des 
%	Seitenspiegels.
%
%-------------------------------------------------------------------------------

\KOMAoption{paper}{a4,portrait}


%-------------------------------------------------------------------------------
% Weitere hier undokumentierte Optionen:
%	Befehl "\pagesize"
%-------------------------------------------------------------------------------


%-------------------------------------------------------------------------------
% Wiederholung der Seitenspiegelberechnung mit allen Angaben
%-------------------------------------------------------------------------------

\KOMAoption{DIV}{last}
