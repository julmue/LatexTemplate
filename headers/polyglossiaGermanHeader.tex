%-------------------------------------------------------------------------------
% package polyglossia:
% * Polyglossia is a package for facilitating multilingual typeseing with 
%	XƎLATEX and (at an early stage) LuaLATEX. Basically, it can be used as a 
%	replacement of babel for performing the following tasks automatically:
%	1.Loading the appropriate hyphenation patterns.
%	2.Setting the script and language tags of the current font (if possible and
%	  available), via the package fontspec.
%	3.Switching to a font assigned by the user to a particular script or 
%	  language.
%	4.Adjusting some typographical conventions according to the current language 
%	  (such as afterindent, frenchindent, spaces before or after punctuation  
%	  marks, etc.).
%	5.Redefining all document strings (like “chapter”, “figure”, 
%	  “bibliography”).
%	6.Adaptingtheformaingofdates(fornon-Gregoriancalendarsviaexternal packages 
%	  bundled with polyglossia: currently the Hebrew, Islamic and Farsi 
%	  calendars are supported).
%	7.For languages that have their own numbering system, modifying the
%	  formatting of numbers appropriately (this also includes redefining the 
%	  alphabetic sequence for non-Latin alphabets).
%	8.Ensuring proper directionality if the document contains languages that are
%	  written from right to left (via the package bidi, available separately).
%-------------------------------------------------------------------------------

\usepackage{polyglossia}
\setdefaultlanguage{german}
